%%%%%%%%%%%%%%%%%%%%%%%%%%%%%%%%%%%%%%%%%
% Simple Academic Report/Thesis Template
% Everything in one file for simplicity
%%%%%%%%%%%%%%%%%%%%%%%%%%%%%%%%%%%%%%%%%

\documentclass[12pt,a4paper]{report}

% ======== Packages ========
\usepackage[utf8]{inputenc}       % Input encoding
\usepackage[T1]{fontenc}          % Font encoding
\usepackage[english]{babel}       % Language
\usepackage{lmodern}              % Latin Modern font
\usepackage{microtype}            % Better typography
\usepackage{graphicx}             % Include graphics
\usepackage{amsmath,amssymb}      % Math symbols
\usepackage{booktabs}             % Professional tables
\usepackage{enumitem}             % Better lists
\usepackage{geometry}             % Page layout
\usepackage{fancyhdr}             % Headers and footers
\usepackage{setspace}             % Line spacing
\usepackage{caption}              % Caption customization
\usepackage{xcolor}               % Colors
\usepackage{listings}             % Code listings
\usepackage{hyperref}             % Hyperlinks - load this last

% ======== Page Setup ========
\geometry{margin=2.5cm}           % Set margins
\setlength{\headheight}{14.5pt}   % Fix for fancyhdr height error
\addtolength{\topmargin}{-2.5pt}  % Compensate for increased header height
\onehalfspacing                   % 1.5 line spacing
\pagestyle{fancy}                 % Use fancy page style
\fancyhf{}                        % Clear header/footer
\fancyhead[L]{\leftmark}          % Chapter name on left
\fancyhead[R]{\thepage}           % Page number on right
\renewcommand{\headrulewidth}{0.4pt} % Header line
\setlength{\parindent}{0pt}       % No paragraph indentation
\setlength{\parskip}{1em}         % Paragraph spacing

% ======== Hyperref Setup ========
\hypersetup{
    colorlinks=true,
    linkcolor=blue,
    citecolor=blue,
    urlcolor=blue
}

% ======== Code Listing Setup ========
\lstset{
    basicstyle=\small\ttfamily,
    breaklines=true,
    commentstyle=\color{green!50!black},
    keywordstyle=\color{blue},
    stringstyle=\color{red},
    frame=single,
    showstringspaces=false
}

% ======== Title Info ========
\title{
    \LARGE\textbf{Project Title}\\
    \vspace{0.5cm}
    \large Final Year Project Report
}

\author{
    Author Name\\
    \vspace{0.3cm}
    Supervisor: Dr. Someone\\
    Department Name\\
    University Name
}

\date{\today}

% ======== Document ========
\begin{document}

% ==== Title Page ====
\maketitle
\thispagestyle{empty}

% ==== Abstract ====
\begin{abstract}
    This document presents the development and implementation of [your project]. 
    The project addresses [problem statement] by [brief description of approach].
    Results demonstrate [brief summary of findings and significance].
\end{abstract}

% ==== Table of Contents ====
\tableofcontents
\clearpage

% ========== Chapter 1: Introduction ==========
\chapter{Introduction}

This chapter introduces the background and objectives of the project.

\section{Background}
Background information about the problem domain and context.

\section{Problem Statement}
Clearly state the problem being addressed in this project.

\section{Objectives}
List and explain the main objectives of the project:

\begin{itemize}
    \item To investigate...
    \item To develop...
    \item To evaluate...
\end{itemize}

\section{Report Structure}
Provide an overview of the structure of this report.

% ========== Chapter 2: State of the Art ==========
\chapter{State of the Art}

A review of related work and technologies relevant to the topic.

\section{Literature Review}
Review of key research papers and publications. For example, Smith et al. \cite{example} proposed...

\section{Existing Solutions}
Analysis of current solutions and their limitations.

\section{Theoretical Background}
Key theories and concepts that underpin this project.

% Example figure placeholder - with improved float positioning
\begin{figure}[htbp]
    \centering
    \framebox{\rule{0cm}{5cm}\rule{10cm}{0cm}}
    \caption{Example figure caption}
    \label{fig:example}
\end{figure}

% ========== Chapter 3: System Design ==========
\chapter{System Design}

Explanation of the architecture, methods, and models used.

\section{System Architecture}
Overview of the system architecture.

\section{Design Decisions}
Explanation of key design decisions and their rationales.

\section{Mathematical Model}
If applicable, present the mathematical model:

\begin{equation}
    f(x) = \sum_{i=1}^{n} w_i \cdot x_i + b
    \label{eq:model}
\end{equation}

% ========== Chapter 4: Implementation ==========
\chapter{Implementation}

Details about the development, tools, and challenges faced.

\section{Development Environment}
Description of tools, libraries, and frameworks used.

\section{Implementation Details}
Technical details of the implementation.

% Example code listing
\begin{lstlisting}[language=Python, caption=Example Python code]
def process_data(data):
    """Process the input data and return results"""
    results = {}
    for item in data:
        results[item.id] = item.value * 2
    return results
\end{lstlisting}

\section{Challenges and Solutions}
Discussion of challenges encountered during implementation and how they were addressed.

% ========== Chapter 5: Results and Analysis ==========
\chapter{Results and Analysis}

Presentation and discussion of results.

\section{Experimental Setup}
Description of the experimental setup and methodology.

\section{Results}
Presentation of results.

% Example table - with improved float positioning
\begin{table}[htbp]
\centering
\caption{Example performance comparison}
\label{tab:results}
\begin{tabular}{@{}lccc@{}}
\toprule
\textbf{Method} & \textbf{Accuracy (\%)} & \textbf{Time (s)} & \textbf{Memory (MB)} \\
\midrule
Baseline & 85.2 & 1.45 & 256 \\
Proposed & 92.7 & 0.95 & 280 \\
\bottomrule
\end{tabular}
\end{table}

\section{Analysis and Discussion}
Analysis and interpretation of the results.

% ========== Chapter 6: Conclusion ==========
\chapter{Conclusion}

Summary of the project, achievements, and future work.

\section{Summary}
Summary of the work done and main findings.

\section{Contributions}
Highlight the main contributions of this work.

\section{Future Work}
Suggestions for future improvements and extensions.

% ==== Bibliography ====
\clearpage
\addcontentsline{toc}{chapter}{Bibliography}
\bibliographystyle{plain}
\bibliography{references}
% Note: You'll need a separate references.bib file or uncomment and use this simple bibliography:
\begin{thebibliography}{9}
\bibitem{example} Smith, J., Johnson, A., "Example Paper Title", Journal of Examples, 2023.
\end{thebibliography}

% ==== Appendices ====
\appendix
\chapter{Additional Material}
Additional information, code listings, or data that support the main text.

\end{document}